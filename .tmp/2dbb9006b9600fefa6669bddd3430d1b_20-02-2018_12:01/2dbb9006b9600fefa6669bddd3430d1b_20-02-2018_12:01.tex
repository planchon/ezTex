% !TEX encoding = UTF-8 Unicode
\documentclass[11pt, a4paper]{article}

\usepackage[T1]{fontenc}
\usepackage[utf8]{inputenc}
\usepackage{lmodern}

\usepackage[top = 50mm, headheight=62pt]{geometry}
\usepackage{titleps}
\usepackage{tabularx, booktabs, multirow}
\usepackage{graphicx, adjustbox, microtype}
\usepackage{lipsum}

\usepackage{amsmath}
\usepackage[linesnumbered,ruled]{algorithm2e}
\usepackage[noend]{algpseudocode}

\makeatletter
\def\BState{\State\hskip-\ALG@thistlm}
\makeatother

\usepackage{multicol}
\usepackage{color}
\usepackage{comment}

\newcommand\doublePoint{::}

\nonstopmode

\usepackage[most]{tcolorbox}

\newtcblisting[auto counter]{texCode}[2][]{
     sharp corners,
     keywordstyle=\color[rgb]{0,0,1},
     fonttitle = \bfseries,
     colframe = gray,
     listing only,
     listing options = {basicstyle = \ttfamily, language = TeX},
     title = Code Source LaTeX - \thetcbcounter : #2, #1
}

\newtcblisting[auto counter]{bashCode}[2][]{
     sharp corners,
     keywordstyle=\color[rgb]{0,0,1},
     fonttitle = \bfseries,
     colframe = gray,
     listing only,
     listing options = {basicstyle = \ttfamily, language = Bash},
     title = Code Source bash - \thetcbcounter : #2, #1
}

\newtcblisting[auto counter]{cppCode}[2][]{
     keywordstyle=\color[rgb]{0,0,1},
     sharp corners,
     fonttitle = \bfseries,
     colframe = gray,
     listing only,
     listing options = {basicstyle = \ttfamily, language = C++},
     title = Code Source C++ - \thetcbcounter : #2, #1
}

\newtcblisting[auto counter]{pytCode}[2][]{
     keywordstyle=\color[rgb]{0,0,1},
     sharp corners,
     fonttitle = \bfseries,
     colframe = gray,
     listing only,
     listing options = {basicstyle = \ttfamily, language = Python},
     title = Code Source Python - \thetcbcounter: #2, #1
}

\newtcblisting[auto counter]{javaCode}[2][]{
     sharp corners,
     keywordstyle=\color[rgb]{0,0,1},
     fonttitle = \bfseries,
     colframe = gray,
     listing only,
     listing options = {basicstyle = \ttfamily, language = Java},
     title = Code Source Java - \thetcbcounter: #2, #1
}

\lstset{%
     numbersep=3mm,
		 numbers=left,
		 numberstyle=\tiny,
		 frame=single,
		 framexleftmargin=6mm,
		 xleftmargin=6mm,
		 literate=%
		  	{doublePoint}{::}2 %
				{code}{CODE}4 %
				{end}{END}3 %
				{endl}{endl}4 %
				{image}{IMAGE}5 %
				{chapter}{CHAPTER}7 %
				% {sub}{SUB_SECTION}9 %
				% {subsub}{SUB_SUB_SECTION}14 %
}

\setlength{\columnseprule}{1pt}
\setlength{\columnsep}{0.5cm}
\def\columnseprulecolor{\color{black}}

\makeatletter
     \renewcommand{\thesection}{\Roman{section}.}
     \renewcommand{\thesubsection}{\Roman{section}.\arabic{subsection}.}
     \renewcommand{\thesubsubsection}{\Roman{section}.\arabic{subsection}.\alph{subsubsection}}
\makeatother

\makeatletter
     \renewcommand{\section}{\@startsection{section}{1}{\z@}%
          {-3.5ex \@plus -1ex \@minus -.2ex}%
          {2.3ex \@plus .2ex}%
          {\reset@font\Large\bfseries	}}
     \renewcommand{\subsection}{\@startsection{subsection}{1}{\z@}%
          {-3.5ex \@plus -1ex \@minus -.2ex}%
          {2.3ex \@plus .2ex}%
          {\reset@font\large\bfseries}}
     \renewcommand{\subsubsection}{\@startsection{subsubsection}{1}{\z@}%
          {-3.5ex \@plus -1ex \@minus -.2ex}%
          {2.3ex \@plus .2ex}%
          {\reset@font\large\bfseries}}
\makeatother

\usepackage{lastpage}

\newpagestyle{style}{
\sethead{}{%
     \begin{tabularx}{\linewidth}[b]{@{}l>{\raggedleft\arraybackslash}X@{}}
          \smash{\raisebox{-0.7\height}{\includegraphics[scale=0.5]{logo.png}}}& \today \\
          \cmidrule[2pt]{2-2}
          & \huge \bfseries Présentation de ezTex \\
          & Rapport de version : alpha 18.2.1 \\
          \addlinespace
          \midrule[0.4pt]
     \end{tabularx}}{}
\setfoot{}{\thepage}{}
}

\usepackage{tocstyle}
\usepackage[nottoc, notlof, notlot]{tocbibind}

\usetocstyle{standard}
\usepackage{mathtools}
\usepackage{amssymb}

\usepackage[pdftex,
            pdfauthor={Paul Planchon},
            pdftitle={Rapport de version : alpha 18.2.1},
            pdfsubject={Présentation de ezTex}, pdfcreator={pdflatex}]{hyperref}
\hypersetup{colorlinks, citecolor=black, filecolor=black, linkcolor=black, urlcolor=black}

\usepackage[french]{babel}

\pagestyle{style}

\begin{document}

\tableofcontents\section{Découvrir \texttt{ezTex} !} 
\subsection{Introduction}Le \LaTeX~ est un peu repoussant pour les débutants, mais il est très interessant de travailler avec car il permet
de faire des documents d'une qualité inégalée. C'est pourquoi j'ai créé ce langage de "programmation" qui vous permet de
generer des documents \LaTeX~ facilement. Vous devez juste apprendre les bases les plus simples de \LaTeX. Le logiciel
s'occupe du reste !
\subsection{L'installer ?}\texttt{ezTex} est très simple à utiliser. Une commande suffit pour créer vos documents, documents d'une qualité exceptionnel (c'est du \LaTeX~ s'il vous plait) !
Pour ce faire, vous devrez installer \texttt{python} et \LaTeX~ biensur ! \newline
\smallskip
Voici comment faire sur Linux-GNU : \newline
\begin{bashCode}{Installation de ezTex} 
>: sudo apt-get install python -y
>: sudo apt-get install texlive-full
\end{bashCode} 
\smallskip
Ces deux commandes vous permettent d'utiliser \texttt{ezTex}.
\newpage
\section{Comment utiliser \texttt{ezTex} ?} 
\subsection{La création et compilation du document \texttt{ezTex}}Utiliser \texttt{ezTex} vous oblige a savoir comment créer et compiler votre document. Dans les prochaines versions une interface graphique (GUI) sera disponible.
Pour l'instant, vous allez devoir créer votre document, il suffit de créer un fichier avec un éditeur de text comme \texttt{atom}, \texttt{geany}, ou encore \texttt{nano} et \texttt{emacs} (pour les plus motivés)
avec l'extension \texttt{.eztex}
La structure d'un document \texttt{ezTex} est très simple :
\smallskip
\begin{bashCode}{Structure de ezTex} 
+- Une "preface"
+- Un corps
   +- Un element de section (chapter)
      +- Du code
	 +- Du texte
	 +- Des images
	 +- ...
   +- Une fin a l'element de section
\end{bashCode} 
\subsubsection{La preface}La préface c'est un peu le cerveau de \texttt{ezTex}, c'est là où vous allez donner toutes les instructions de compilation. Cette partie permet aussi de définir la titre, sous-titre, date et auteur.
Vous disposez de plusieurs élements pour la personnaliser : \newline
\begin{itemize}
\item \texttt{TITLE} : vous permet de définir le titre
\item \texttt{SUB\_TITLE} : vous permet de définir le sous-titre
\item \texttt{AUTHOR} : vous permet de définir l'auteur
\item \texttt{TABLE\_OF\_CONTENT} : active (si \texttt{TRUE}) ou désactive la table des matières
\item \texttt{SAVE} : garde le fichier compilé dans le dossier \texttt{\/render}
\item \texttt{BACK\_UP} : Fait une copie à chaque compilations (si l'ancienne compilation est plus vielle d'une minute).
\item \texttt{PREVIEW} : active la prévisualisation
\end{itemize}
\newpage
\subsubsection{Le corps et ses élements}Une fois la préface configurée, vous allez pouvoir créer votre document. Sachez pour commencer que \texttt{toutes} les commandes \LaTeX~ sont disponibles.
Ainsi, si \texttt{ezTex} ne propose pas une fonctionnalitée, vous pouvez faire ce que vous voulez en utilisant \LaTeX. \newline
Les commandes que \texttt{ezTex} propose sont :
\begin{bashCode}{L'environnement d'organisation} 
chapter doublePoint TITRE
	(text, algo, images ...)
	SUB_SECTION doublePoint TITRE
	SUB_SUB_SECTION doublePoint TITRE
end
\end{bashCode} 
Vous pouvez organiser vos documents en chapitre, sous-section et sous-sous-section. Les formats par défault de \LaTeX sont aussi supportés, mais
une belle intégration dans le document \texttt{ezTex} n'est pas garantie. \newline
\begin{bashCode}{L'environnement code} 
code doublePoint LANGAGE doublePoint TITRE
	Du code ici, indentation
		comprise
end
\end{bashCode} 
Les langages supportés sont : \texttt{java, c++, tex \& bash}. Plus a venir dans le futur ! \newline
\begin{bashCode}{L'environnement image} 
image doublePoint LIEN IMAGE doublePoint ARG TAILLE doublePoint TITRE
\end{bashCode} 
Cet environment permet d'afficher des images dans votre document. Elles seront centrés. La taille est personnnalisable a votre gise. Si vous voulez une autre
disposition, vous pouvez utiliser l'environment que propose \LaTeX. \newline
Un environment "algo" va faire apparition dans les prochaines versions. Il vous permettra de créer des algorithmes en pseudo code d'une façon extremement simple !
\subsection{La compilation}Pour l'instant la compilation se fait par terminal. Dans les prochaines versions une interface graphique sera disponible. Utilisez la commande suivante :
\begin{bashCode}{Commande de compilation} 
python genTex.py FICHER\_A\_COMPILER
\end{bashCode} 
\newpage
\section{Exemples d'utilisation :} 
\subsection{Exemples de code source}\begin{bashCode}{Hello-world en C++} 
code doublePoint C++ doublePoint Hello world !
	int main(){
		cout << "Hello world!" << endl;
	}
end
\end{bashCode} 
Se compilera pour donner :
\begin{cppCode}{Hello world !} 
int main(){
	cout << "Hello world!" << endl;
}
\end{cppCode} 
Nota Bene : L'indentation n'est pas supportée pas \texttt{ezTex}. Vous devez indenter vous-même. \texttt{ezTex} ne détruiera pas l'indentation cependant.
Le logiciel la comprendra et la transcrira. \newline
\subsection{Exemples d'intégration d'images}\begin{bashCode}{Intégration d'images} 
image doublePoint meme.jpg doublePoint scale = 0.4 doublePoint COMMENT ?
\end{bashCode} 
Nota Bene : Les images doivent etre dans le même dossier que le fichier \texttt{ezTex}. \newline
Se compilera pour donner :
\begin{figure}[h!] 
\centerline{\fbox{\includegraphics[scale = 0.4]{meme.jpg}}} 
\caption{COMMENT ?} 
\end{figure}\smallskip


\end{document}
